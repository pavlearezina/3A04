\documentclass[12pt, titlepage]{article}
\usepackage{geometry}
\usepackage{changepage}
\usepackage{enumerate}
\title{SE 3A04: Software Design III: Large System Design}
\author{Group \#5, Spaceship System Sabatoge %Alliteration; always adored and absolutely amazing
		\\Pareek Ravi 001407109
		\\Pavle Arezina 001410366
		\\David Hobson 001412317
		\\Victoria Graff 001401451
		\\Julian Cassano 001406891
}
\date{\today}
\begin{document}
\maketitle
\pagenumbering{roman}
\tableofcontents
\listoftables
\listoffigures
\begin{table}[bp]
\caption{\bf Revision History}
\end{table}
\newpage
\pagenumbering{arabic}

\section{Overall Description}
\subsection{Product Perspective}
Spaceship System Sabotage is an independent and self-containted system. The typical user experience with this application is to keep the system operating for as long as possible so that after a specified period of time the space craft will "reach its intended destination" and the user will reach an end state through a win condition. This perspective on the interaction of sub-systems will make the application feel more like a game than it does a simulation. When a sub-system is inputted with a stimulus to cause it to fail, the application will tell the user that one of the crew members of the ship has sabotaged a sub-system. This acts as a challenge for the user to keep the crew members alive but attempt to determine who the defector of the crew is, providing "lore" and context for the simulated sub-system failure so that the user has motivation to keep the system functional.
%do a block diagram, use examples from xa3 srs

\subsection{Product Functions}
\begin{itemize}
\item The user will be able to start a simulation of a system
\item The different sub-systems will be displayed for the user
\item The user will be able to switch viewpoints to choose the sub-system they are observing
\item Messages/alerts will be displayed for the user
\item The sub-systems will be able to receive stimulus from the application to prompt a status change
\item The sub-systems will be able to affect other sub-systems depending on their status
\item The user will be able to input a stimulus that will have a specified effect on a sub-system, namely a status change back to the original state
\item The simulation will be able to reach an end state at any time and terminate the simulation
\item The user will be able to end the simulation at any time if they so choose
\end{itemize}
%they want more graphics so draw them a picture of a pony or some shit idk

\subsection{User Characteristics}
The intended user of this application is anyone who enjoys games or challenges and has a basic understanding of systems and how they can interact. No previous work or school experience with designing systems is required as this application is not a test on the user's understanding of software architecture. This application aims to provide a casual experience of simulating the negative effects of sub-systems failing in a larger system and allows the user to have a role in this scenario regardless of failure or not. Each end state is an acceptable level of completion in terms of the simulation as they both result in a demonstration of the interaction between sub-systems. This means that the application is accessible to anyone from casual players in elementary school to Dr. Khedri while he's killing time waiting for his TAs to mark these deliverables.

\subsection{Constraints}
The limitations of this project come in the form of development time. The allocated amount of hours that we intend to dedicate to this project are in negative correlation with the amount of other school work we have to do during the timeframes of each deliverable. We would love to make this a full fleshed out game that is fun and challenging but at the moment we do not have the time to comfortably meet the requirements of the project and introduce enough additional features to make this a fun to play, sellable game. The rubric specified by the TAs will first and foremost be followed, and only then can additional features be added.
%add another section that says something smart about the language we are using once we determine that

\subsection{Assumptions and Dependencies}
% I need to see requirements before I do this *COUGH* PAREEK AND DAVID *COUGH*

\subsection{Apportioning of Requirements}
%In version 1.1 we'll include a game mode where the creators of FTL sue us

\section{Non-Functional Requirements}
\label{sec:non-functional_requirements}
% Begin Section
\subsection{Look and Feel Requirements}
\label{sub:look_and_feel_requirements}
% Begin SubSection

\subsubsection{Appearance Requirements}
\label{ssub:appearance_requirements}
% Begin SubSubSection
\begin{enumerate}[{LF}1. ]
	\item The application shall have a clean, minimalistic user interface. The systems on the ship will be colour coded and easy to differentiate on the map.
\end{enumerate}
% End SubSubSection

\subsubsection{Style Requirements}
\label{ssub:style_requirements}
% Begin SubSubSection
\begin{enumerate}[{LF}1. ]
	\item The application shall have smooth transitions while responding to user input. 
\end{enumerate}
% End SubSubSection

% End SubSection

\subsection{Usability and Humanity Requirements}
\label{sub:usability_and_humanity_requirements}
% Begin SubSection

\subsubsection{Ease of Use Requirements}
\label{ssub:ease_of_use_requirements}
% Begin SubSubSection
\begin{enumerate}[{UH}1. ]
	\item The application shall be simple to use for all users above the age of twelve. A new user shall be able to navigate through menus and play the game with ease.
\end{enumerate}
% End SubSubSection

\subsubsection{Personalization and Internationalization Requirements}
\label{ssub:personalization_and_internationalization_requirements}
% Begin SubSubSection
\begin{enumerate}[{UH}1. ]
	\item There will be no personalization or Internationalization requirements.
\end{enumerate}
% End SubSubSection

\subsubsection{Learning Requirements}
\label{ssub:learning_requirements}
% Begin SubSubSection
\begin{enumerate}[{UH}1. ]
	\item The learning time within the application shall take no longer than five minutes.
\end{enumerate}
% End SubSubSection

\subsubsection{Understandability and Politeness Requirements}
\label{ssub:understandability_and_politeness_requirements}
% Begin SubSubSection
\begin{enumerate}[{UH}1. ]
	\item The application will be easy to understand and will refrain from inappropriate language or slang words.
\end{enumerate}
% End SubSubSection

\subsubsection{Accessibility Requirements}
\label{ssub:accessibility_requirements}
% Begin SubSubSection
\begin{enumerate}[{UH}1. ]
	\item The application will be playable for the hearing impaired.
\end{enumerate}
% End SubSubSection

% End SubSection

\subsection{Performance Requirements}
\label{sub:performance_requirements}
% Begin SubSection

\subsubsection{Speed and Latency Requirements}
\label{ssub:speed_and_latency_requirements}
% Begin SubSubSection
\begin{enumerate}[{PR}1. ]
	\item The application shall update systems from user input in under 5 seconds, and load time between screens shall be under 5 second.
\end{enumerate}
% End SubSubSection

\subsubsection{Safety-Critical Requirements}
\label{ssub:safety_critical_requirements}
% Begin SubSubSection
\begin{enumerate}[{PR}1. ]
	\item There will be no Safety-critical requirements.
\end{enumerate}
% End SubSubSection

\subsubsection{Precision or Accuracy Requirements}
\label{ssub:precision_or_accuracy_requirements}
% Begin SubSubSection
\begin{enumerate}[{PR}1. ]
	\item The touch user input on the application will have 90 percent accuracy.
\end{enumerate}
% End SubSubSection

\subsubsection{Reliability and Availability Requirements}
\label{ssub:reliability_and_availability_requirements}
% Begin SubSubSection
\begin{enumerate}[{PR}1. ]
	\item The application shall be available at all times.
\end{enumerate}
% End SubSubSection

\subsubsection{Robustness or Fault-Tolerance Requirements}
\label{ssub:robustness_or_fault_tolerance_requirements}
% Begin SubSubSection
\begin{enumerate}[{PR}1. ]
	\item The application will be designed to handle all selection errors by the user.
\end{enumerate}
% End SubSubSection

\subsubsection{Capacity Requirements}
\label{ssub:capacity_requirements}
% Begin SubSubSection
\begin{enumerate}[{PR}1. ]
	\item The application will be able to handle input from one user.
\end{enumerate}
% End SubSubSection

\subsubsection{Scalability or Extensibility Requirements}
\label{ssub:scalability_or_extensibility_requirements}
% Begin SubSubSection
\begin{enumerate}[{PR}1. ]
	\item The application will be created to be able to integrate more space ship systems in the future.
\end{enumerate}
% End SubSubSection

\subsubsection{Longevity Requirements}
\label{ssub:longevity_requirements}
% Begin SubSubSection
\begin{enumerate}[{PR}1. ]
	\item The application will operate as long as the Android OS version it is created on, is supported.
\end{enumerate}
% End SubSubSection

% End SubSection

\subsection{Operational and Environmental Requirements}
\label{sub:operational_and_environmental_requirements}
% Begin SubSection

\subsubsection{Expected Physical Environment}
\label{ssub:expected_physical_environment}
% Begin SubSubSection
\begin{enumerate}[{OE}1. ]
	\item The application will run on Android smartphones.
	\item The application shall be used at any location.
\end{enumerate}
% End SubSubSection

\subsubsection{Requirements for Interfacing with Adjacent Systems}
\label{ssub:requirements_for_interfacing_with_adjacent_systems}
% Begin SubSubSection
\begin{enumerate}[{OE}1. ]
	\item There will be no adjacent systems used through this application.
\end{enumerate}
% End SubSubSection

\subsubsection{Productization Requirements}
\label{ssub:productization_requirements}
% Begin SubSubSection
\begin{enumerate}[{OE}1. ]
	\item There will be no productization requirements for this application.
\end{enumerate}
% End SubSubSection

\subsubsection{Release Requirements}
\label{ssub:release_requirements}
% Begin SubSubSection
\begin{enumerate}[{OE}1. ]
	\item The application will be downloadable through the Google Play Application store.
\end{enumerate}
% End SubSubSection

% End SubSection

\subsection{Maintainability and Support Requirements}
\label{sub:maintainability_and_support_requirements}
% Begin SubSection

\subsubsection{Maintenance Requirements}
\label{ssub:maintenance_requirements}
% Begin SubSubSection
\begin{enumerate}[{MS}1. ]
	\item The application shall be built through modular design for easy maintenance and for updating features in the future.
\end{enumerate}
% End SubSubSection

\subsubsection{Supportability Requirements}
\label{ssub:supportability_requirements}
% Begin SubSubSection
\begin{enumerate}[{MS}1. ]
	\item The application will only run on android smartphones.
\end{enumerate}
% End SubSubSection

\subsubsection{Adaptability Requirements}
\label{ssub:adaptability_requirements}
% Begin SubSubSection
\begin{enumerate}[{MS}1. ]
	\item The application will be designed to be able to re-format to fit all android phone screen sizes.
\end{enumerate}
% End SubSubSection

% End SubSection

\subsection{Security Requirements}
\label{sub:security_requirements}
% Begin SubSection

\subsubsection{Access Requirements}
\label{ssub:access_requirements}
% Begin SubSubSection
\begin{enumerate}[{SR}1. ]
	\item The application will be availability to users that own an android smartphone.
\end{enumerate}
% End SubSubSection

\subsubsection{Integrity Requirements}
\label{ssub:integrity_requirements}
% Begin SubSubSection
\begin{enumerate}[{SR}1. ]
	\item No personal information will be required.
\end{enumerate}
% End SubSubSection

\subsubsection{Privacy Requirements}
\label{ssub:privacy_requirements}
% Begin SubSubSection
\begin{enumerate}[{SR}1. ]
	\item No personal information will be required.
\end{enumerate}
% End SubSubSection

\subsubsection{Audit Requirements}
\label{ssub:audit_requirements}
% Begin SubSubSection
\begin{enumerate}[{SR}1. ]
	\item There will be no Audit requirements for this application
\end{enumerate}
% End SubSubSection

\subsubsection{Immunity Requirements}
\label{ssub:immunity_requirements}
% Begin SubSubSection
\begin{enumerate}[{SR}1. ]
	\item There will be no immunity requirements for this application.
\end{enumerate}
% End SubSubSection

% End SubSection

\subsection{Cultural and Political Requirements}
\label{sub:cultural_and_political_requirements}
% Begin SubSection

\subsubsection{Cultural Requirements}
\label{ssub:cultural_requirements}
% Begin SubSubSection
\begin{enumerate}[{CP}1. ]
	\item The application will display language in the english language and will avoid mention or assumption of any specific culture.
\end{enumerate}
% End SubSubSection

\subsubsection{Political Requirements}
\label{ssub:political_requirements}
% Begin SubSubSection
\begin{enumerate}[{CP}1. ]
	\item The application will not mention any political topics or issues.
\end{enumerate}
% End SubSubSection

% End SubSection

\subsection{Legal Requirements}
\label{sub:legal_requirements}
% Begin SubSection

\subsubsection{Compliance Requirements}
\label{ssub:compliance_requirements}
% Begin SubSubSection
\begin{enumerate}[{LR}1. ]
	\item The application shall adhere to all compliance laws in Ontario, Canada.
\end{enumerate}
% End SubSubSection

\subsubsection{Standards Requirements}
\label{ssub:standards_requirements}
% Begin SubSubSection
\begin{enumerate}[{LR}1. ]
	\item The application shall adhere to all standards laws in Ontario, Canada.
\end{enumerate}
% End SubSubSection

% End SubSection

% End Section

\end{document}